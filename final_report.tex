% !TEX TS-program = pdflatex
% !TEX encoding = UTF-8 Unicode

% This is a simple template for a LaTeX document using the "article" class.
% See "book", "report", "letter" for other types of document.

\documentclass[11pt]{article} % use larger type; default would be 10pt

\usepackage[utf8]{inputenc} % set input encoding (not needed with XeLaTeX)

\usepackage{amssymb,amsmath}

%%% Examples of Article customizations
% These packages are optional, depending whether you want the features they provide.
% See the LaTeX Companion or other references for full information.

%%% PAGE DIMENSIONS
\usepackage{geometry} % to change the page dimensions
\geometry{a4paper} % or letterpaper (US) or a5paper or....
% \geometry{margin=2in} % for example, change the margins to 2 inches all round
% \geometry{landscape} % set up the page for landscape
%   read geometry.pdf for detailed page layout information

\usepackage{graphicx} % support the \includegraphics command and options

% \usepackage[parfill]{parskip} % Activate to begin paragraphs with an empty line rather than an indent

%%% PACKAGES
\usepackage{booktabs} % for much better looking tables
\usepackage{array} % for better arrays (eg matrices) in maths
\usepackage{paralist} % very flexible & customisable lists (eg. enumerate/itemize, etc.)
\usepackage{verbatim} % adds environment for commenting out blocks of text & for better verbatim
\usepackage{subfig} % make it possible to include more than one captioned figure/table in a single float
% These packages are all incorporated in the memoir class to one degree or another...

%%% HEADERS & FOOTERS
\usepackage{fancyhdr} % This should be set AFTER setting up the page geometry
\pagestyle{fancy} % options: empty , plain , fancy
\renewcommand{\headrulewidth}{0pt} % customise the layout...
\lhead{}\chead{}\rhead{}
\lfoot{}\cfoot{\thepage}\rfoot{}

%%% SECTION TITLE APPEARANCE
\usepackage{sectsty}
\allsectionsfont{\sffamily\mdseries\upshape} % (See the fntguide.pdf for font help)
% (This matches ConTeXt defaults)

%%% ToC (table of contents) APPEARANCE
\usepackage[nottoc,notlof,notlot]{tocbibind} % Put the bibliography in the ToC
\usepackage[titles,subfigure]{tocloft} % Alter the style of the Table of Contents
\renewcommand{\cftsecfont}{\rmfamily\mdseries\upshape}
\renewcommand{\cftsecpagefont}{\rmfamily\mdseries\upshape} % No bold!


%%% END Article customizations

%%% The "real" document content comes below...

\title{Parallel Work}
\author{Aaron Mininger and James Kirk}
%\date{} % Activate to display a given date or no date (if empty),
         % otherwise the current date is printed 



\begin{document}
\maketitle

\section{Introduction}
This paper focuses on the parallelization of episodic memory; particularly the
episodic memory mechanism in Soar: a general cognitive architecture used and
developed at Michigan by John Laird and his students. The goal of Soar is to
support the capabilities required of a general intelligence agent by providing
mechanisms for that agent to use. One of those mechanisms is episodic memory,
which captures snapshots or episodes of the agent’s state (working memory)
through time and stores them for later retrievals. The agent can then construct
a cue to use to search through episodic memory and find an episode that matches.

It is important for the agent to be reactive, which places strict constraints on
efficiency in any memories utilized. The current implementation of Soar’s
episodic memory has a number of efficient ways of performing a search. However,
in the worst case (the cue does not have a match or only a match in the oldest
episode) a search must be done through all of the episodes, which can be quite a
large number depending on the lifetime of the agent.

\subsection{Episodic Memory in Soar}
Working memory is represented as a directed (possibly cyclic) graph. It consists
of a list of working memory elements (wme’s) that represent an edge in the graph
(starting node, edge name, ending node). During the storage of the current state
of working memory as an episode, the entire graph is not saved. Instead, only
the changes from the last episode (additions or deletions) are tracked and
stored. This means that random access into the past is not a cheap operation, as
the episode needs to be reconstructed from the data.

The two main operations of Soar’s episodic memory mechanism are storing new
episodes and retrieving the best match for a given cue. The following sections
explain in some detail how these processes were made efficient in the existing
serial version.

\subsection{Storage}
Episodic memory is stored inside an sqlite database. As mentioned before, the
working memory graph is represented as a list of edges, or wme’s. Throughout the
lifetime of the agent, individual wme’s may be added and removed a number of
times. What is stored in the database is a list of all the wme’s that have
existed in the agent’s past, plus the intervals for which they were active. The
agent determines which working memory elements have changed since the last
storage, and either ends an active interval (for deletions), or begins a new one
(for additions). This process is very efficient, since the algorithm only has to
consider currently active wme’s; the number of which remains relatively small
and constant over the agent’s lifetime.

\subsection {Retrieval}
The agent initiates a retrieval by creating a cue to search with. The cue
represents an acyclic subgraph of working memory and the goal is to find the
episode that matches that subgraph the best. The agent does a walk backwards
from the current episode and only searches those episodes that changed with
regards to the cue. It first does a surface match based on the leaves of the
cue. The leaves are considered independent, so the structure as a whole is not
unified to the episode being considered. If it finds a matching containing all
of the leaves it does a full graph match. This is expensive (NP-complete) so it
is only done when absolutely necessary.

Whenever it finds a graph match it stops the search and returns the result. This
is because results are biased by recency; if two episodes are a perfect match
the more recent one is returned. If no episode perfectly matches, then the
entire contents of memory end up being searched. Partial matches are ranked and
the highest one is returned. The ranking is calculated from the number of leaves
that match and is again biased by recency.

\section{Parallel Implementation}
Our goal in parallelizing Soar’s episodic memory was to distribute the storage
and search of episodic memory across a set of processors. Although the search
process has been heavily optimized, it still grows over time with the number of
episodes. Agents that run over hours and days are of interest to study, and
eventually the cost of using episodic memory grows too large for the agent to be
reactive. Our goal is to be able to reduce that cost by spreading it out over a
number of processors. This can then extend the time that long-lived agents can
remain reactive.

We did not set out to change the behavior of episodic memory or re-implement the
algorithms underlying the storage and retrievals. These have already been
heavily optimized and tuned for performance and any work on our part would have
had marginal effects.




The goal of the parallel implementation is to not change the behavior of
episodic memory, while paralleling the episodic search process. We have elected
to implemented a Distributed memory system, as sharing the large database would
cause lots of access problems. Each processor must therefore also store its own
data to search through.

The basic parallel structure of the implementation requires at least 3
processors. The first processor is the Agent processor, which handles the Soar
agent running process. The second processor is a Manager processor that handles
queries and episode addition requests from the agent process and disperses the
work to Worker processors, of which there must be at least one. The Manager must
also arbitrate over the results it gets back from the workers. The following
sections describe how the episodes are stored and retrieved on each worker.

\subsection{Storage}

Kernel modifications and Database allocation

Adding new episodes

Each worker holds a section of episodes. This size is determined by the window
size of each episode. This window size can be adjusted per worker dynamically as
the agent accumulates new episodes and conducts queries of episodic memory. This
division of episodes, and the number of episodes allowed on each worker, load
balances the work to be done. When a worker hits its cap of episodes, it will
remove its oldest episode from its graph and database to send to its immediate
neighbor. The neighbor will add the episode. Again what is being sent is not
actually episodes, but diff structure contain the nodes and edges of the graph
that were added or removed.

The last worker in the chain will not remove any episodes. The earlier workers
contain the most recent episodes. For example, running with 8 processors,
processor 0 is the agent, processor 1 is the manager, processor 2 receives the
newest episodes from the manager, and processor 7 contains the oldest episodes.

When the agent creates another episode this new episode is messaged to the
Manager. The Manager notifies all worker processors that it has received a new
episode. If the Worker has an episode to send, meaning that it contains at least
as many episodes as its window size, it will immediately remove the oldest
episode to send to its neighbor. The first worker processor then receives the
new episode from the Manager. It is possible that a worker that removed and sent
an episode will not actually receive a new one, temporarily under filling its
episode allocation by 1.

This prevents a linear problem that occurred on the first implementation of
episode addition. If a worker waits to see if he will receive a new episode from
his neighbor, you have to wait potentially O(p) time, where p is the number of
processors, for the episode addition to propagate to the last worker. The
process to remove the oldest episode is the most expensive computation wise in
the storage of new episodes, so this linear delay significantly slowed down the
storage process of episodes.


\subsection{Retrieval}
When the agent wants to retrieve an episode it sends this request and the cue in
a message to the Manager. The Manager broadcasts the search request message to
all worker processes. Once a worker receives a search request it performs an
identical search algorithm as done in the serial version, only on its partition
of the episode data that it has. Each episode will return its best match, if it
has a match, in a message to the manager.

The manager selects the best episode matching from the responses. If it receives
a response that is better than any it could possibly receive from the workers it
is still waiting on, such as getting a full graph match on the first worker, the
manager will immediately send this response back to the agent. As is done in the
serial version, the manager selects the best match based on, in order of
importance, full graph matching, the match score, and if a tie recency. The
division of episodes insures that if a worker processor has a lower id than
another, it episodes will always be more recent.

\subsection{Dynamic Load Balancing}

The window size of each worker determines the overall load balancing of the
retrieval process. The window size does not need to be fixed, and as we are
continuously adding episodes, we can increase the window size for each agent to
scale with the number of episodes. There are a few different window size scaling
strategies that could be beneficial for different agent behaviors.

The most basic scaling strategy just keeps an even split of the data among the
processors. If all episodes will need to be searched, as in the worst case, then
this will be the best load balancing. Based on the number of episodes a worker
has received it can calculate when it needs to increase its window size without
messaging any other processors. The equation for determining if the window size
needs to be increased is shown below in equation 1.

$incrWindow? = ((count \%  2^(P_i) -1 ) == 0)$
Where count is the total number of episodes this worker has received, P is the
number of processors, and i is the id of the worker processor. Every time the
worker has received another $ 2^(P_i) -1 $ episodes it should increment its window
size by 1.

Another strategy is to have the size of the window increase exponential from the
first worker to the last worker, which holds the oldest episodes. For example
for a small case with 5 workers and 310 episodes added so far, the workers would
hold in order for youngest to oldest, 10, 20, 40, 80, and 160 episodes. This
compares to the above strategy where each worker would hold 62 episodes. If the
agent is always retrieving recent episode this will likely be an improvement
from the previous strategy. In this case if the matching episode is in the first
70 episodes, the exponential window size strategy will likely finish before an
even window strategy. Similar to the equation for the previous strategy, the
worker can determine when to increase its window size. (equation 2)

$incrWindow? = ((count \%  P_i ) == 0)$

However, the idea of Soar is to have general purpose mechanisms like episodic
memory that are useful for agents. When developing an agent, there should not be
required changes to the kernel to select the best episodic memory behavior.
Additionally we should not require agent developers to characterize their
episodic memory retrievals, whether they are mostly in recent memory or older
memory or some mix. Even with a given agent this behavior could differ depending
on the situation the agent is in.

To address this situation a third strategy was developed to split the difference
between the above strategies and to dynamically alter behavior based on
retrievals. The difference between above strategies number of episodes received
before a window size increase, $2^(P_i) -1$ in the first case and $P_i$ in the
second case. In the dynamic version we select a value between these two by means
of a split variable between 0 and 1.0. The equations for this algorithm can be
seen below.

$mod = (2^(P_i) -1)*split + (P_i) *(1-split) incrWindow? = ((count \%  mod ) == 0)$

The value of split is initially set to 0.5, splitting the difference evenly so
that the episode allocation is not even, but also less than exponential in
increase. Every time a retrieval occurs, the worker episode that returned the
best result is used to update the split value, a global value the is calculated
by the manager and broadcast to all workers. An update split value can be
calculated by the percentile that the best response came from. If the best
response comes from the earliest worker, processor 2, the update value will be
1, and if from the oldest worker the update value will be 0. The split value is
averaged with this update value using a tuning parameter. If the parameter is
0.5 then the update value contributes to half of the new split value. In
experimentation this parameter was varied over a few different values to see the
effect on window size growth and distribution.

\section{Experimental Results}
\subsection{Evaluation Specifics}

To evaluate the new episodic memory a variety of agents with different epmem
behaviors were tested. The number of episodes, number of processors, and size
and complexity of cues were all varied to analyze the timing behavior of the new
code, compared against the serial version. The results were evaluated with the
different strategies for load balancing episodes among the worker processors as
well. It is important not only to analyze the timing differences in retrievals,
but also in the process of the parallelized episode storage.

There were three different types of agents that were created to tests the
different versions of the program. The first agent was designed to always do
worst case retrievals. A worst case retrieval occurs when there is no complete
graph match and all episodes must be searched to determine best score, or when
the graph match is on the earliest episode. The latter condition was implemented
for the worst case agent. The second agent always was designed to retrieve
recent episodes. The agent will issue a cue that will have a match in the most
recent 10% of episode history. The agent will randomly generate a cue that will
match an episode anywhere between 0 and 10% of the episode history. The final
agent issues random retrievals that should have a complete graph match, but will
not be in a designated area of the episode history.

For all of the agents, the retrievals and episode creation were setup as
follows. At every time step the agent adds and removes elements from working
memory, which will cause the addition of new episodes to episodic memory. Every
1,000 times steps, or 1,000 episodes, the agent issues an epmem query. Depending
on the agent this will be a long retrieval, a recent retrieval, or a random
retrieval. The time is captured from right before the agent sends the query
message, to when the agent receives the response. The serial timing is captured
similarly, starting after the construction of the cue, and ending when a
response is found.

As well as doing a retrieval every 1000 time steps, the timing of episode
storage is recorded every 1001 time steps. The offset is used so that a good
representation of episode addition times is collected. We noticed during testing
that some parallel episode storages take significantly longer than others. This
is due to some storages not requiring any old episode removals, ie every worker
is not full or not receiving a new episode. When there are only 2 workers, every
other episode addition is quicker. The offset ensures that we get an even
sampling of the addition times for comparison.


\subsection{Data Results}

\section{Analysis}

\section{Conclusions}

\section{References}

Derbinsky, N., and Laird, J. E. (2009). Efficiently Implementing Episodic
Memory. Proceedings of the International Conference on Case-based Reasoning.


\subsection {Subsection}


\end{document}